\section{Background}

\subsection{Intents} When it comes to Intents there is a distinction between explicit and implicit types of intents \cite{intent}.
An explicit Intent is primarily used for launching internal activities since it carries specific information as to what class is to be put on the activity stack and executed. The Intent can be explicitly run via the startActivity() method providing only the launching context and the target class to be executed as constructor parameters.


{\footnotesize\begin{lstlisting}
@Override
public void onClick(View arg0)
{
    startActivity(new Intent(FirstActivity.this, 
    	SecondActivity.class));
}
\end{lstlisting}}

The use of implicit Intents complicates the matter a bit, in that the level of abstraction becomes higher. The Intent is no longer directly associated to an activity but rather a generic action that later, as a result of Intent Resolution, will be mapped to a specific activity or service.

{\footnotesize\begin{lstlisting}
Intent intent = new Intent(Intent.ACTION_SEND);
intent.setType("text/plain");
intent.putExtra(android.content.Intent.EXTRA_TEXT, 
	"News for you!");
startActivity(intent);
\end{lstlisting}}


Since more than one activity or service can be eligible for carrying out a generic action, the Manifest-file of the application will define the conditions upon which to choose the correct activity or service given the context. When an activity is declared in the manifest the use of Intent Filters serve to inform which implicit actions they can handle.

{\footnotesize\begin{lstlisting}
<intent-filter . . . >
   <action android:name="com.example.project.SHOW_CURRENT" />
   <action android:name="com.example.project.SHOW_RECENT" />
   <action android:name="com.example.project.SHOW_PENDING" />
   . . .
</intent-filter>
\end{lstlisting}}

Further implicit intent distinctions can be made by declaring categories as sub-elements to filters.

While the explicit intents are fairly trivial the implicit intents can grow to be rather complex not least in the defining of proper filters and categories. If at the end of the intent resolution more than one suitable activity has been found for carrying out the implicit intent the user will be prompted to decide which activity will be allowed to proceed with the action. 


\subsection{Tools}
\textbf{EMF.} Eclipse Modelling Framework (EMF) is a meta-model framework. A self-describing meta-model in EMF is called Ecore. Any EMF model has tree-like structure, where there is a root element and other elements are contained by the root explicitly or through other elements. 


\textbf{XText.}


\textbf{JDT.}
