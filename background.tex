\section{Background}
\label{background}

\subsection{Intents} 
\label{intents}
Intents can activate three of the components of an application: activities, services and broadcast receivers.
When it comes to Intents there is a distinction between explicit and implicit types of intents \cite{intent}.
An explicit Intent is primarily used for launching internal activities since it carries specific information as to what class is to be put on the activity stack and executed. The code for this type of an Intent is presented in a Listing \ref{explicitIntent}. The Intent can be explicitly run via the startActivity() method; the system then receives this call to start a new instance of the Activity requiring only the launching context and the target class to be executed as constructor parameters.

\footnotesize\begin{lstlisting}[label=explicitIntent,caption=Explicit Intent]
@Override
public void onClick(View arg0){
    Intent i = new Intent(this, SecondActivity.class);
    startActivity(i, 1);
}
\end{lstlisting}

The use of implicit Intents (listing \ref{implicitIntent}) complicates the matter a bit, in that the level of abstraction becomes higher. The Intent is no longer directly associated to an activity but rather a generic action that later, as a result of Intent Resolution, will be mapped to a specific activity or service.

{\footnotesize\begin{lstlisting}[label=implicitIntent,caption=Implicit Intent]
Intent intent = new Intent(Intent.ACTION_SEND);
intent.setType("text/plain");
intent.putExtra(android.content.Intent.EXTRA_TEXT, 
	"News for you!");
startActivity(intent);
\end{lstlisting}}

In the case where an Intent requires a callback function, the Intent must be started with the function onStartActivityResult(). This function takes an additional integer parameter that is used in the callback function to determine the activity which is being returned.

Activities have a single focus point, interaction with users. Activities class takes care of which windows those the application is placed in the UI. 


Since more than one activity or service can be eligible for carrying out a generic action, the Manifest-file of the application will define the conditions upon which to choose the correct activity or service given the context. When an activity is declared in the manifest the use of Intent Filters serve to inform which implicit actions they can handle. Examples of Intent Filters are presented in a listing \ref{intentFilters}.

{\footnotesize\begin{lstlisting}[label=intentFilters,caption=Intent Filters]
<intent-filter . . . >
   <action android:name="com.example.project.SHOW_CURRENT" />
   <action android:name="com.example.project.SHOW_RECENT" />
   <action android:name="com.example.project.SHOW_PENDING" />
   . . .
</intent-filter>
\end{lstlisting}}

Further implicit intent distinctions can be made by declaring categories and data as sub-elements to filters. A category contains additional information about the kind of component that should handle the Intent. An Intent can pass the category test when every category in the Intent object must match a category in the filter. A data element can specify a URI (scheme, host, port, and path) and a data type (MIME type). For the data test both the URI and the data type in the Intent object are compared to a URI and data type specified in the filter. 

While the explicit intents are fairly trivial the implicit intents can grow to be rather complex not least in the defining of proper filters and categories. If at the end of the intent resolution more than one suitable activity has been found for carrying out the implicit intent the user will be prompted to decide which activity will be allowed to proceed with the action. 


\subsection{Tools}
\label{tools}
\textbf{EMF. Ecore.} Eclipse Modelling Framework (EMF) \cite{emf} is a meta-model framework for modelling and code generating purposes. EMF allows building tools and other applications based on a data model. A self-describing meta-model in EMF is called Ecore. Any EMF model has tree-like structure with a root element and other elements are contained by the root itself or through other elements. 

\textbf{Xtext.} Xtext \cite{xtext} is a framework for development of programming languages and domain specific languages. It can generate a parser and a fully featured Eclipse-based IDE. It is possible to write a grammar to specify a language, this process is done and written in Xtext's grammar language. This grammar describes how an Ecore model is extracted from a textual notation (a code generator derives an ANTLR parser and the classes for the object model).

\textbf{Eclipse JDT and Eclipse AST.} Eclipse Java Development Tool (JDT) \cite{jdt} provide the API for accessing and manipulating the Java source code by adding plug-ins. Eclipse Abstract Syntax Tree (AST) \cite{ast} is a detailed tree representation of the Java source code. This tree is easier and more convenient to analyze and modify programmatically than text-based source. AST provides an API for easy changing, adding, deleting and reading the source code. Basically each Java code file is presented as a subclass of ASTNode, that provides specific information about the object it represents. Every subclass is specialized for an element of the Java Programming Language like: expressions, names, statements, types and type body declarations and they contain specific information for each element. For instance a type body declaration will contain information about the name, return type, parameters, etc.

