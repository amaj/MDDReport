\section{Conclusion}
\label{conclusion}

Our project demonstrates an effective use of Model Driven Development to build systems using multiple layers of abstraction. While the scope of our project may be too small to show its applicability in the industry, the techniques employed have proven to effectively increase the productivity of a software development process when a basis using MDD is established.
With the MDD approach taken during this project process we have addressed the shortcomings of normal high-level languages in respect to excluding platform-specificity and rather introduce domain specific concepts as a means of development. 

As previously mentioned in Section \ref{threatstovalidity} - Threats to Validility, the only weak point of our final system is the size of the database used, and that all special cases when working with Intents on the Android platform may not have been fully explored. Automated tests with testing frameworks such as JUnit should also be implemented to ensure correctness over varied scenarios.

The evaluation of the plug-in clearly shows a significant decrease in time taken to develop an error free application when using the plug-in. We can conclude that the plug-in we have created fulfills all of our initial requirements, meets our targeted goals, and also includes some additional functionality that was not set in the base requirements.

The final code can be found on GitHub \footnote{GitHub: https://github.com/jycr753/IntentCore}, and the working plug-in can be installed into Eclipse from via the "Install New Software" component using URL found in footnotes \footnote{http://thinklehost.com/plugin} (ensure "Group items by category" is unchecked).


