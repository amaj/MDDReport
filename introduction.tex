\section{Introduction}

An Android application can contain zero or more activities. When the application has some activities needed to be run, it is important to be able to transit from one activity to another to perform another task either with or without information from the first activity. In Android, the navigation between activities is possible thanks to Intents.

Intents are asynchronous messages which provide a facility for performing late runtime binding between the code in different applications and allow Android components to request functionality from other components of the Android system. Intents can also be used to signal to the Android system that a certain event has occurred.

Basically an Intent is a passive data structure holding an abstract description of an action to be performed. For instance, lots of Android applications allow their users to share some data with other people e.g. via Twitter application. It is possible to send data to one of this components through an Intent.

\subsection{Motivation}
Motivation of the project is to make implementing Intent while developing Android applications easier for the programmer. We want to create a user friendly Eclipse plug-in which allows an user to choose an intent from the intents list and to put the generated code of chosen intent with default parameters settings into a proper place in the code. The code can be modified by an user e.g. it is possible to change the values of parameters.

\subsection{State of the art}
An example of a current technology for dealing with Model-to-Text transformation is the open source Acceleo project \cite{acceleo}, development of which started four years ago. The project aims to ease and speed up the writing of tedious framework specific completion code, and is able to create code generators specific to several frameworks, like Android for one.

The Acceleo project provides an abstract syntax for generating concrete code and offers editor features such as highlighting, content assistance and error detection. The code generator works using a template file written in the Acceleo syntax, that reads in a model and generates simple .java files. The code generator can be further customized by specifying blocks in the template that must be implemented by the user post generation.

An additional example of Model-to-Text transformation that is specifically built for Android devices is the project Gplad (Graphical Programming Language for Android Devices) \cite{gplad}. It is a domain-specific language that allows code generation in a block-based programming interface, to obtain solutions in various common programming languages, including Java. It's aim is provide a simple programming environment without necessarily needing advanced coding knowledge.

\subsection{Problem Description} 
The specific problem we aim to solve surrounds decreasing development errors, and increasing productivity.

The code required to initialize Intents, whilst relatively simple in its basic form, is prone to producing errors because of the customizability and extensive optional parameters and uses. An Intent call in a simple case will use just a few lines of code, whereas a fully extended and defined Intent could, through using additional data and specific parameter settings, grow to be quite verbose.

\subsection{Goals}
Our goal is to develop a DSL and prototype an Eclipse plug-in that generates Android Intent implementation code through a Model-to-Model transformation from our DSL to the Eclipse editor AST. Our ultimate aim in this project is to counter the problematic areas previously identified, namely; decreasing overall development time, and reducing the number of development errors. The developed plug-in should enable us to effectively achieve our goals through an easy to use, and clear interface.

The outcome of achieving our goals is to demonstrate the effective use of Model Driven Development in an area which has previously had little research with MDD. This demonstration should show how a relatively simple implementation of an MDD project could be extended to cover wider and more complex situations.

\subsection{Methodology}
The first stage in our development is to analyse the DSL anatomy of Android Intents and the related sections, this DSL will then be developed into an EMF Ecore Model. Once the Model has been created, a selection of Android Intents will be chosen as test subjects and will be built using as Dynamic Instances through the EMF tools. This relatively small subset of Intents will be chosen for their common usage and variance, and will allow us to revise the model and ensure it is capable of handling the different variety of Intents possible.

Once our Model has been revised to an extent that we believe it is suitable for the majority of available Intents, we will then develop our Xtext grammar using the generated grammar that Xtext provides as a base. The grammar will be revised to simplify the development in the concrete syntax, and our previous selection of Intents will be built in this Xtext grammar.

As proof of concept, will then develop a simple Model-to-Text (M2T) transformation using Xpand that will allow us to generate simple Java files which will demonstrate that our models are capable of producing the code required. The final step in development is to build the Eclipse plug-in that reads our models and generates the Eclipse AST nodes required to provide a Model-to-Model transformation. The final stage is to throughly test our plug-in and run our evaluation.

If time allows, additional functionality will be built into the plug-in which will manage the Android Manifest XML file permission nodes, provide exception handling code, and finally provide additional support for individual Intent cases where the standard code can be enhanced to give better support (for example, Intents that require a response require an additional method to receive this response).

This planned methodology enables us to revise our DSL and model several times through development, and ensure each stage is completed successfully, but these additional processes and iterations, will increase development time.