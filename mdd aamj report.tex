\documentclass{llncs}

\usepackage[english]{babel}
\usepackage[utf8x]{inputenc}
\usepackage{amsmath}
\usepackage{graphicx}
\usepackage{listings}
\usepackage{float}
\usepackage{natbib}
\usepackage{url}
\usepackage{color}

\definecolor{dkgreen}{rgb}{0,0.6,0}
\definecolor{gray}{rgb}{0.5,0.5,0.5}
\definecolor{mauve}{rgb}{0.58,0,0.82}

\lstset{frame=tb,
  language=Java,
  aboveskip=3mm,
  belowskip=3mm,
  showstringspaces=false,
  columns=flexible,
  basicstyle={\small\ttfamily},
  numbers=none,
  numberstyle=\tiny\color{gray},
  keywordstyle=\color{blue},
  commentstyle=\color{dkgreen},
  stringstyle=\color{mauve},
  breaklines=true,
  breakatwhitespace=true
  tabsize=3
}

\title{Generating Android Intent implementation code using EMF and DSL}
\author{Ashley Davison-White, ashw@itu.dk 
      \\Agnieszka Majkowska, amaj@itu.dk
      \\Jorge Y. Castillo Rodríguez, jyur@itu.dk
      \\Marrtin Rylander, mryl@itu.dk}


\institute{IT University of Copenhagen}



\date{\today}

\begin{document}
\maketitle
\thispagestyle{plain}
\pagestyle{headings}
\setcounter{page}{1}
\pagenumbering{arabic}

 \begin{abstract}
This paper describes the process of creating a development tool for programmers using the Android platform, specifically in Intent code generation. Intents are abstract ways of describing processes to be performed, and the implementation code can be verbose and tedious to write. While the coding of Intents is not difficult to comprehend, the specifics of data types and settings for each Intent makes the process error prone. The tool we will be describing in this paper is to be seen as a case study into the MDD approach for creating a prototype that allows a programmer to insert generated code for an Intent call at a desired location in the Eclipse editor. The prototype comes in the form of a plug-in to be installed in Eclipse, which is generated based on a Domain Specific Language created using Ecore EMF models and Xtext. A Model-to-Model transformation of models developed in the DSL into models in Eclipse's Java Abstract Syntax Tree (AST) is performed using the Eclipse Java Development Tools (JDT). After conducting a proper trial testing the efficacy of using our plug-in the results were clearly indicating a remarkable cut in average implementation time.   
\\[1\baselineskip]
\textbf{Keywords.} M2M, DSL, EMF, Ecore, M2M, Xtext, Android, Intent
 \end{abstract}

\section{Introduction}

Mobile applications are increasingly interwoven into one another, and applications are able to use a plethora of different applications available to perform a required task. This could be a range of tasks; from viewing a video, processing a payment, sharing content through social networks - the list is virtually endless. Android applications perform this transition from one application to another through a system known as Intents; and an Android application can contain zero or more Intent calls.

Intents are asynchronous messages which provide a facility for performing late runtime binding between the code in different applications, which allows Android components to request functionality from other components of the Android system. Data can be passed to and can be received from a calling Intent, and they can also be used to signal to the Android system that a certain event has occurred. Essentially, an Intent is a passive data structure, holding an abstract description of an action to be performed. 

The code required to initialize Intents, whilst relatively simple in its basic form, is prone to producing errors because of the customizability and extensive optional parameters and uses. An Intent call in a simple case will use just a few lines of code, whereas a fully extended and defined Intent could, through using additional data and specific parameter settings, grow to be quite verbose. 

The goal and motivation of the project is to demonstrate an effective use of Model Driven Development methodology, and evaluate the usefulness of the MDD approach to development. We will do this by developing an Eclipse plug-in that will make implementing Intents in development of Android applications easier for programmers. The plug-in should be a simple, user-friendly GUI driven view that will allow a user to choose an Intent from a list, and generate the code required to call the chosen Intent in the correct format, and place it in the editor window. All available settings for the chosen Intent will be generated, and this will enable the user to modify the generated code to suit their needs quicker and with less errors when compared to manually implementing the code. This demonstration should show how a relatively simple implementation of an MDD project could be extended to cover wider and more complex situations.

An example of a current technology for dealing with Model-to-Text transformation is the open source Acceleo project \cite{acceleo}, development of which started four years ago. The project aims to ease and speed up the writing of tedious framework specific completion code, and is able to create code generators specific to several frameworks, like Android for one.

The Acceleo project provides an abstract syntax for generating concrete code and offers editor features such as highlighting, content assistance and error detection. The code generator works using a template file written in the Acceleo syntax, that reads in a model and generates simple .java files. The code generator can be further customized by specifying blocks in the template that must be implemented by the user post generation.

An additional example of Model-to-Text transformation that is specifically built for Android devices is the project Gplad (Graphical Programming Language for Android Devices) \cite{gplad}. It is a domain-specific language that allows code generation in a block-based programming interface, to obtain solutions in various common programming languages, including Java. It's aim is provide a simple programming environment without necessarily needing advanced coding knowledge.

The first stage in our development is to analyze the DSL anatomy of Android Intents and the related sections, this DSL will then be developed into an EMF Ecore Model. Once the Model has been created, a selection of Android Intents will be chosen as test subjects and will be built using as Dynamic Instances through the EMF tools. This relatively small subset of Intents will be chosen for their common usage and variance, and will allow us to revise the model and ensure it is capable of handling the different variety of Intents possible.

Once our Model has been revised to an extent that we believe it is suitable for the majority of available Intents, we will then develop our Xtext grammar using the generated grammar that Xtext provides as a base. The grammar will be revised to simplify the development in the concrete syntax, and our previous selection of Intents will be built in this Xtext grammar.

The next step in development is to build the Eclipse plug-in that reads our models and generates the Eclipse AST nodes required to provide a Model-to-Model transformation. The final stage is to thoroughly test our plug-in and run our evaluation.

This planned methodology enables us to revise our DSL and model several times through development, and ensure each stage is completed successfully, but these additional processes and iterations, will increase development time.

The work we have done over the course of this project has concluded with a final plug-in for which the scope of functionality can be further expanded, with our current code base providing the basis for further development into areas such as:

\begin{itemize}
\item Code generation to a greater extent. This could include generation of code for broadcast receivers and Intent callback methods.
\item Automatic code completion.
\item Extracting Intent calls from existing code.
\item Acquiring the data for Intent code generation, from an external database.
\item Adding permissions to the manifest file automatically, if the data we have concerning a specific Intent indicates that it is required.
\end{itemize}

This paper is organized as follows. In Section \ref{background} we present the Android Intent syntax and make an overview of the tools we use in the project. In Section \ref{realisation} we present the step-by-step solution for the Intent code generation. In Section \ref{evaluation} we evaluate our solution, and then discuss its results and the threats to validity of our solution in Section \ref{threatstovalidity}. In section \ref{usecases} we present some use cases. In section \ref{relatedwork} we present some of the found related work. In Section \ref{Discussion}, we suggest ideas for future projects, and finally we conclude our research in Section \ref{conclusion}.

\section{Background}
\label{background}

\subsection{Intents} 
\label{intents}
When it comes to Intents there is a distinction between explicit and implicit types of intents \cite{intent}.
An explicit Intent is primarily used for launching internal activities since it carries specific information as to what class is to be put on the activity stack and executed. The code for this type of an Intent is presented in a Listing \ref{explicitIntent}. The Intent can be explicitly run via the startActivity() method; the systems then receives this call to start a new instance of the Activity requiring only the launching context and the target class to be executed as constructor parameters.

\footnotesize\begin{lstlisting}[label=explicitIntent,caption=Explicit Intent]
@Override
public void onClick(View arg0){
    Intent i = new Intent(this, SecondActivity.class);
    startActivity(i, 1);
}
\end{lstlisting}

The use of implicit Intents (listing \ref{implicitIntent}) complicates the matter a bit, in that the level of abstraction becomes higher. The Intent is no longer directly associated to an activity but rather a generic action that later, as a result of Intent Resolution, will be mapped to a specific activity or service.

{\footnotesize\begin{lstlisting}[label=implicitIntent,caption=Implicit Intent]
Intent intent = new Intent(Intent.ACTION_SEND);
intent.setType("text/plain");
intent.putExtra(android.content.Intent.EXTRA_TEXT, 
	"News for you!");
startActivity(intent);
\end{lstlisting}}

In the case where an Intent requires a callback function, the Intent must be started with the function onStartActivityResult(). This function takes an additional integer parameter that is used in the callback function to determine the activity which is being returned.

Activities have a single focus point, interaction with users. Activities class takes care of which windows those the application is placed in the UI. 


Since more than one activity or service can be eligible for carrying out a generic action, the Manifest-file of the application will define the conditions upon which to choose the correct activity or service given the context. When an activity is declared in the manifest the use of Intent Filters serve to inform which implicit actions they can handle. Examples of Intent Filters are presented in a listing \ref{intentFilters}.

{\footnotesize\begin{lstlisting}[label=intentFilters,caption=Intent Filters]
<intent-filter . . . >
   <action android:name="com.example.project.SHOW_CURRENT" />
   <action android:name="com.example.project.SHOW_RECENT" />
   <action android:name="com.example.project.SHOW_PENDING" />
   . . .
</intent-filter>
\end{lstlisting}}

Further implicit intent distinctions can be made by declaring categories as sub-elements to filters.

While the explicit intents are fairly trivial the implicit intents can grow to be rather complex not least in the defining of proper filters and categories. If at the end of the intent resolution more than one suitable activity has been found for carrying out the implicit intent the user will be prompted to decide which activity will be allowed to proceed with the action. 


\subsection{Tools}
\label{tools}
\textbf{EMF. Ecore.} Eclipse Modelling Framework (EMF) \cite{emf} is a meta-model framework for modeling and code generating purposes. EMF facilitates building tools and other applications based on a structured data model. A self-describing meta-model in EMF is called Ecore. Any EMF model has tree-like structure, where there is a root element and other elements are contained by the root explicitly or through other elements. 

\textbf{Xtext.} Xtext \cite{xtext} is a framework for development of programming languages and domain specific languages. It can generate a parser and also a class model for the Abstract Syntax Tree and a fully featured, customizable Eclipse-based IDE. It is possible to write a grammar to specify a language, this process is done and written in Xtext's grammar language. This grammar describes how an Ecore model is derived from a textual notation. From that definition, a code generator derives an ANTLR parser and the classes for the object model.

\textbf{Eclipse JDT and Eclipse AST.} Eclipse Java Development Tool (JDT) \cite{jdt} provide the API for accessing and manipulating the Java source code. The JDT tool contributes a set of plug-ins that add the capabilities of a full-featured Java IDE to the Eclipse platform. Eclipse Abstract Syntax Tree (AST) \cite{ast} is a detailed tree representation of the Java source code. This tree is more convenient and reliable to analyse and modify programmatically than text-based source. AST provides an API for easy changing, adding, deleting and read the source code. Basically each java code file is presented as a subclass of ASTNode, which provides specific information about the object it represents. Every subclass is specialized for an element of the Java Programming Language like: expressions, names, statements, types and type body declarations. Every subclass of ASTNode contains specific information for the Java element it represents for instance a type body declaration will contain information about the name, return type, parameters, etc.



\section{Realisation}

\begin{figure}[h!]
  \centering
    \includegraphics[width=\textwidth]{metamodel}
  \caption{Meta-model of Intent}
\end{figure}

\begin{figure}[h!]
  \centering
    \includegraphics[width=\textwidth]{codegenerator}
  \caption{Code generator view}
\end{figure}

\section{Evaluation}
\subsection{Method}
To evaluate our plug-in we want to conduct a proper trial to highlight the benefits of using the code generation from the plug-in which carries out the Intent instantiation for you as opposed to having developers write the code themselves.

We will base our test primarily on measuring the difference in time spent creating a group of intents to be inserted in a pre-made Android application with just one Activity in it. The method stubs are left empty for the trial participants to fill out with the correct code, using either the plug-in or manually. We were briefly contemplating having trial participants building the entire application from scratch but regressed from this idea in that a trial design like that would be introducing to many variables to our test and it would be difficult to dissect the results and clearly establish the effects of using our plug-in.

The intents required for the participants to create also requires them to insert permissions into the application manifest file, which the plug-in handles for the participants who are instructed to use it.

The participants were split into two groups based on Java programming experience. The least experienced were instructed to use the plug-in and the more experienced were told to do the implementation on their own, with the Google search-engine as their only help.

For the trial we wanted to focus on getting quantifiable data which is why we decided to measure the implementation time, so we can not know for sure if using our plug-in would bring about qualitative improvements as well.

The participants were given a short introduction to the task first, and they were given opportunity to see the application in action on an Android device. After this the subjects were placed in front of a computer and the timer was started. Deciding when the subject was done the task was done by reviewing the code they had written.


\subsection{Results}
We conducted the trial and recorded the results given in table below:
 

\begin{tabular}{|c|c|c|}
\hline 
& With the plug-in & Without the plug-in \\ 
\hline 
& 1:20 & 2:59 \\ 
\hline 
& 1:08 & 7:02 \\ 
\hline 
& 1:30 & 4:20 \\ 
\hline 
& 2:00 & 3:30 \\ 
\hline 
 & 1:09 & • \\ 
\hline
Average: & 1:25 & 4:28
\end{tabular} 

As is palpably shown from the table the average implementation time was cut severely by a factor of ~3, suggesting that the use of the plug-in may very well contribute to the effectiveness of the developer.

These result were very much along the lines of what we had hoped for when we initially set out to create the prototype, and the vast gap in implementation time between the two groups clearly indicates that the use of code generation tools may be of use not only to cut the time spent typing and making documentation lookups but also to further the correctness of what is written and reduce time spent bug-fixing as a project develops. (AM I LAYING IT ON TO THICK HERE????)  











\section{Threats to validity}

\subsection{Internal Threats}
There are several possible problems that can occur when relying on black box testing to ensure the validity of software. The primary concern is ensuring all possible paths through the software have been tested, whether user interaction has avoided possible errors, or simply ignored them, and ensuring all functionality of the software has been tested thoroughly. Some of these possible error situations could be discovered and resolved through automated testing, but by basing our project on existing software with a long history and extensive testing suits, it can be strongly argued for the integrity of our code by ensuring it conforms to all standards.

Our testing method could also be improved upon by including more participants, and increase the amount of testing each participant is required to perform. This would help reduce the amount of potential anomalies in our results, and ensure better test coverage for our plug-in. Our testing would also ideally include experienced Android developers with enough knowledge to be able to write Intent code without the need to look up variables and code structures; this would put a new element into our testing so that we could compare the time taken to write code, against the time taken to open our plug-in, find the correct Intent action, and ensure that the code generated is correctly inserted.

However, this threat against our plug-in can be extinguished by referring to our original aim, which was to reduce development errors and time taken for developers, by showing that our test results clearly indicate that non-experienced Java developers using our plug-in were significantly faster at developing our test application against experienced Java developers without it.

A final internal threat to our plugin is the lack of data that our plug-in holds. Although our dataset covers a wide range of Intents, it is relatively small and therefore we have only tested a very limited set of Intents through our development. It therefore not possible to confirm that our plug-in would work for all of the many Intents available.

\subsection{External Threats}

External threats to our plug-in are limited but potentially drastic. Without the ability to update the Intent database that our plug-in stores, we cannot ensure that our database has up-to-date and correct information for the all intents available. This could cause potential problems if the Android core Intents were to change in future versions. It is also true that our current dataset cannot guarantee compatibility across all versions of the Android SDK.

This version compatibility issue also goes for the Eclipse JDT library, although changes to a core component of Eclipse are unlikely and would most probably be backwards compatible.

\section{Use cases}
\label{usecases}

\label{Use cases}
In the 1994 thesis by John Edward Hutchinson titled "An Empirical Assessment of Model Driven Development in Industry", the author brings forth the stories of how a handful of companies in the industry managed to implement an MDD approach to their software development, of particular interest was that of an international Printer Company.
The products of The Printer Company(company names are kept anonymous) are intricate puzzles of mechanical, electrical and software engineering where the embedded software constituted a bottleneck in the development process. The Company slowly introduced MDD on a per project basis with an initial pilot project leading the way for the rest of the product line to undergo same transformation. 
The company's results were astounding in that it led to a reduction in need of man hours but also led to a discovery of greater commonality among their systems, which became the basis for an explicit reuse strategy. After adopting an MDD approach the company was able shorten the time to market and increase the quality assurance.
Similar success stories are reported by other companies in the car industry and telecom industry where the businesses main line of products involve both the mechanical and/or electrical and software engineering disciplines. The use of MDD allows for the programmers to focus on more advanced parts of the development process, while the software base is maintained using an abstract language(DSL) that in and of itself does not require the developer to have an extensive engineering background, but maybe more of a domain expert instead.
This improves the cost-effectiveness and the speed of development, which makes it understandable why companies in the industry are taking an interest to MDD when looking to further their ability to compete in the market.


\section{Related Work}
\label{relatedwork}

\textbf{Model2Text Transformation.} There are plenty of articles presenting different approaches to Model2Text transformation, however we can only highlight few of them because of space limits.

Albert, Munoz, Pelechano and Pastor show in the their article \cite{manoli} a Model2Text transformation where UML association specification are transformed into C\# code. For the project purpose it was necessary to extend UML proposal with association relationships and to create an input models by using a conceptual framework. In addition project required to define the set of rules to generate the C\# code. Model2Text transformation was implemented in EMD and MOFScript tools.

Ugaz in his article \cite{ugaz} presents the process of Model2Text transformation using the Epsilon Generation Language (EGL), the DSLs are implemented in MetaDepth. EGL transforms a model created in DSML and formalism into the code. The formalism is the one of Role-Playing Games. The target text is a code in Java for Android application framework. The author focuses on 3 fundamental DSM elements: a domain-specific language, a domain-specific code generator and a domain-specific framework.


\textbf{Text2Model Transformation.} Breslav focused in his paper \cite{breslav} on creating textual syntax for Domain-Specific Languages (DSL). The main concept is to represent analysis of textual syntax as a sequence of transformations, which is made by using abstract syntax trees (ASTs). the author divided the transformation process into 2 parts: Text2AST which is handled by openArchitectureWave and AST2Model proposed by the author.

\textbf{Model2Model Transformation.} In the paper \cite{atl} the authors present ATL (ATLAS Transformation Language) that is a domain-specific language that is designed to solve common Model2Model transformation tasks. ATL is a hybrid transformation language, it has declarative and imperative constructs. A module which is a transformation definition in ATL, contains a header section (with a name of the module, source and target models), import section, helpers (values are specified by OCL expression) and transformation rules. The authors present also the ATL development tools that are built on top of Eclipse platform. They allow to perform the major tasks in usage of any programming language like compiling, executing, editing and debugging.


\textbf{FSML.} Antkiewicz and Czarnecki wrote an interesting paper concerning FSML \cite{FSML}, where they present the concept of FSMLs with round-trip engineering support. They focus on few challenges related to this topic: knowing how to write framework completion code, viewing the design of the completion code and the migration of the code to the new framework API versions. Framework-Specific Modeling Language (FSML) is a special category of Domain-Specific Language (DSL) that is defined on top of an object-oriented application framework, they model abstractions and rules of application programming interfaces (APIs). FSMLs help developers understand, analyze, create, migrate and evolve application code by showing how applications use APIs. FSMLs are an explicit representation of the domain-specific concepts provided by framework APIs. FSMLs are used for expressing framework-specific models of application code, which describe instances of framework-provided concepts that are implemented in the application code. In an FSML each concept instance is characterized by a configuration of features, which represents implementation steps or choices. FSML concept configuration describes how the framework should be completed in order to create the implementation of the concept (how the concept should be implemented in the code). Such models may be connected with the application code through a forward (the generation of code from FSMs by successively executing transformations for code pattern addition), a reverse (the automatic retrieval of FSMs from application code by detecting feature instances in the code) mapping enabling round-trip engineering (RTE).



\section{Discusion}
\label{Discusion}

Attractive innovation that could be added into the already existing eclipse plug-in are as follows:

- \textbf{Intent dependencies and options.} Adding additional information to the interface such as code previews, descriptions, a list of required permissions, a list of any required dependencies, and possibly links to documentation, would assist the developer during development. Whilst we currently store permissions and callbacks in our database, the additional information would require modifications to the meta-model and DSL. This information would be included alongside the below list in the plug-in interface.

\begin{figure}[H]
\label{codegeneratorview}
  \centering
    \includegraphics[width=\textwidth]{intentBefore}
  \caption{Future Work 1.0}
\end{figure}

- \textbf{Creating constants and variables}. Many Intents require additional data to perform the call correctly and currently the code generation only inserts pre-defined information as a helper for the developer. A better way of setting this information would be through an interface requesting the user enter the information through the plug-in interface, most probably in a dialog window. This would also resolve the issue that our callback code generation currently has with the variable integer required to detect which call is being returned.

- \textbf{API level support}. Detecting the Android SDK version used, and then modifying the available Intents and code generated for the particular SDK would be a complicated but good feature. This would ensure the code generation works across as many platforms as possible, but compiling the data for each SDK version would be a long and extensive process.

- \textbf{Web-service updates}. Updating the database via a web-service API would ensure the database stays up-to-date and would remove the need to publish new versions of the plug-in when the database should be updated. This would require we setup a server which contains the file but the existing could would remain relatively unchanged, with the only difference being the database file that is loaded.
	

\section{Conclusion}
\label{conclusion}

Our project demonstrates an effective use of Model Driven Development to build systems using multiple layers of abstraction. While the scope of our project may be too small to show its applicability in the industry, the techniques employed have proven to effectively increase the productivity of a software development process when a basis using MDD is established.
With the MDD approach taken during this project process we have addressed the shortcomings of normal high-level languages in respect to excluding platform-specificity and rather introduce domain specific concepts as a means of development. 

As previously mentioned in the Threats to Validity section, the only weak point of our final system is the size of the database used, and that all special cases when working with Intents on the Android platform may not have been fully explored. Automated tests with testing frameworks such as JUnit should also be implemented to ensure correctness over varied scenarios.

The evaluation of the plug-in clearly shows a significant decrease in time taken to develop an error free application when using the plug-in. We can conclude that the plug-in we have created fulfils all of our initial requirements, meets our targeted goals, and also includes some additional functionality that was not set in the base requirements.

\renewcommand{\bibname}{References}
\setlength{\bibsep}{0.0pt}
\bibliographystyle{plain}
\bibliography{references}

\end{document}