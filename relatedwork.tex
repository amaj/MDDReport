\section{Related Work}
\label{relatedwork}
Whilst we did go through several papers during the project course, these papers are the ones we've considered to be the most relevant.

\textbf{Model2Model Transformation.} In the paper \cite{atl} the authors present ATL (ATLAS Transformation Language) which is a domain-specific language that is designed to solve common Model-to-Model transformation tasks. ATL is a hybrid transformation language, it has declarative and imperative constructs. A module that is a transformation definition in ATL, contains a header section (with a name of the module, source and target models), import sections, helpers (values are specified by OCL expressions) and transformation rules. The authors also present the ATL development tools that are built on top of Eclipse platform which allow developers to perform major tasks, such as compiling, executing, editing and debugging.


\textbf{FSML.} Antkiewicz and Czarnecki wrote an interesting paper concerning FSML \cite{FSML}, where they present the concept of FSMLs with round-trip engineering support. They focus on few challenges related to this topic: knowing how to write framework completion code, viewing the design of the completion code and the migration of the code to the new framework API versions. Framework-Specific Modelling Language (FSML) is a special category of Domain Specific Language (DSL) that is defined on top of an object-oriented application framework, they model abstractions and rules of application programming interfaces (APIs). FSMLs help developers understand, analyze, create, migrate, and evolve application code by showing how applications use APIs. FSMLs are used for expressing framework-specific models of application code, that describe instances of framework-provided concepts that are implemented in the application code. In an FSML, each concept instance is characterized by a configuration of features, which represents implementation steps or choices. FSML concept configuration describes how the framework should be completed in order to create the implementation of the concept (how the concept should be implemented in the code). Such models may be connected with the application code through a forward (the generation of code from FSMs by successively executing transformations for code pattern addition), or a reverse (the automatic retrieval of FSMs from application code by detecting feature instances in the code) mapping - enabling round-trip engineering (RTE). It is possible to implement our plug-in as an FSM, but since our project is less generic and more specific, the increased development time would not outweigh the benefits provided.
