\section{Use cases}
\label{usecases}

\label{Use cases}
In the 1994 thesis by John Edward Hutchinson titled "An Empirical Assessment of Model Driven Development in Industry", the author brings forth the stories of how a handful of companies in the industry managed to implement an MDD approach to their software development, of particular interest was that of an international Printer Company.
The products of The Printer Company(company names are kept anonymous) are intricate puzzles of mechanical, electrical and software engineering where the embedded software constituted a bottleneck in the development process. The Company slowly introduced MDD on a per project basis with an initial pilot project leading the way for the rest of the product line to undergo same transformation. 
The company's results were astounding in that it led to a reduction in need of man hours but also led to a discovery of greater commonality among their systems, which became the basis for an explicit reuse strategy. After adopting an MDD approach the company was able shorten the time to market and increase the quality assurance.
Similar success stories are reported by other companies in the car industry and telecom industry where the businesses main line of products involve both the mechanical and/or electrical and software engineering disciplines. The use of MDD allows for the programmers to focus on more advanced parts of the development process, while the software base is maintained using an abstract language(DSL) that in and of itself does not require the developer to have an extensive engineering background, but maybe more of a domain expert instead.
This improves the cost-effectiveness and the speed of development, which makes it understandable why companies in the industry are taking an interest to MDD when looking to further their ability to compete in the market.
